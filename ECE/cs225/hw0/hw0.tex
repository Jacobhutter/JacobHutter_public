\documentclass[11pt]{article}
\usepackage{jeffe,handout}
%\def\rmdefault{bch} % Use Charter for main text font.

\def\BOX#1{\fbox{\vbox to #1{\vss\hbox to #1{\hss}}}}
\def\Bigbox{\BOX{0.25in}}
\def\Bigbox{\raisebox{-0.5ex}[0.25in][0pt]{\BOX{0.25in}}}

\hidesolutions

\renewcommand{\arraystretch}{2}

% =========================================================
\begin{document}

\headers{CS 225}{ }{Fall 2016}

\begin{center}
    \LARGE
    \textbf{HWK 0}
    \\[1ex]
    \Large Due August 29, 2016 \\
    in lecture and SVN \\
    \large Instructions for submission into your \\
    class SVN repository are on the webpage.
\end{center}

\bigskip\hrule
\begin{quote}
    The purpose of this assignment is to give you a chance to refresh the math
    skills we expect you to have learned in prior classes. These particular
    skills will be essential to mastery of CS225, and we are unlikely to take
    much class time reminding you how to solve similar problems. Though you are
    not required to work independently on this assignment, we encourage you to
    do so because we think it may help you diagnose and remedy some things you
    might otherwise find difficult later on in the course. If this homework is
    difficult, please consider completing the discrete math requirement (CS173
    or MATH 213) before taking CS225.

    \bigskip
    \hrule
\end{quote}

\begin{table}[h]
    \centering
    \renewcommand{\arraystretch}{1.5}
    \begin{tabular}{ll}
        \textbf{Name}: & \\
        \textbf{NetID}: & \\
        \textbf{Section} (circle one):
        & \parbox[t]{4.5in}{
            \begin{tabular}[t]{llll}
                Wednesday & 7--9pm  AYB &             &            \\ \hline
                Thursday  & 9--11am AYC & 11--1pm AYD & 1--3pm AYE \\
                          & 3--5pm  AYF & 5--7pm  AYG & 7--9pm AYH \\ \hline
                Friday    & 9--11am AYI  & 1--3pm AYK & 3--5pm AYL \\
                          &  5--7pm  AYM &  \\ \hline\hline
                \multicolumn3l{\bfseries Laptop Sections:} \\ \hline
                Thursday  & 9--11am AYS & 11--1pm  AYN & 1--3pm  AYO \\
                          & 3--5pm AYP & 5--7pm AYT \\ \hline
                Friday    & 9--11am AYU & 1--3pm AYQ & 3--5pm AYR \\
                          & 5--7pm AYV \\ \hline
            \end{tabular} \\
        }
    \end{tabular}
\end{table}

\vfil
\begin{table}[h]
    \centering
    \begin{tabular}{|l|l|c|} \hline
        Grade & \hspace{.75in} & Out of 60 \\ \hline
        Grader & \multicolumn{2}{c|}{} \\ \hline
    \end{tabular}
\end{table}

\newpage
\begin{problems}
%----------------------------------------------------------------------
\item (3 points)
    Using 140 characters or less, post a synopsis of your favorite movie to the
    course piazza space under the ``HW0 tell me something!'' notice, so that
    your post is visible to everyone in the class, and tagged by \#HW0num1.
    Also, use Piazza's code-formatting tools to write a {\em private} post to
    course staff that includes at least 5 lines of code. It can be code of your
    own or from a favorite project---it doesn't even have to be syntactically
    correct---but it must be formatted as a code block in your post, and also
    include the tag \#HW0num1. (Hint: Check
    \url{http://support.piazza.com/customer/portal/articles/1774756-code-blocking}).
    Finally, please write the 2 post numbers corresponding to your posts here:

    \begin{table}[h]
        \begin{center}
            \begin{tabular}{|l|c|}
                \hline
                Favorite Movie Post (Public) number: & \hspace{2in} \\ \hline
                Formatted Code Post (Private) number: & \hspace{2in} \\ \hline
            \end{tabular}
        \end{center}
    \end{table}

\item (12 points)
    Simplify the following expressions as much as possible, \textbf{without
    using an calculator (either hardware or software)}. Do not approximate.
    Express all rational numbers as improper fractions. Show your work in the
    space provided, and write your answer in the box provided.

    \begin{enumerate}
        \item $\displaystyle\prod_{k=2}^n \left(1-\frac{1}{k^2}\right)$
                \hfill
                \begin{tabular}{|l|c|}
                    \hline
                    Answer for (\theenumii): & \hspace{2in} \\ \hline
                \end{tabular}
                \vfill

        \item $\displaystyle3^{1000} \bmod 7$
                \hfill
                \begin{tabular}{|l|c|}
                    \hline
                    Answer for (\theenumii): & \hspace{2in} \\ \hline
                \end{tabular}
                \vfill
\newpage

        \item $\displaystyle\sum_{r=1}^\infty(\frac{1}{2})^r$
                \hfill
                \begin{tabular}{|l|c|}
                    \hline
                    Answer for (\theenumii): & \hspace{2in} \\ \hline
                \end{tabular}
                \vfill

        \item $\displaystyle\frac{\log_7 81}{\log_7 9}$
                \hfill
                \begin{tabular}{|l|c|}
                    \hline
                    Answer for (\theenumii): & \hspace{2in} \\ \hline
                \end{tabular}
                \vfill

        \item $\displaystyle\log_2 4^{2n}$
                \hfill
                \begin{tabular}{|l|c|}
                    \hline
                    Answer for (\theenumii): & \hspace{2in} \\ \hline
                \end{tabular}
                \vfill

        \item $\displaystyle\log_{17} 221 - \log_{17} 13$
                \hfill
                \begin{tabular}{|l|c|}
                    \hline
                    Answer for (\theenumii): & \hspace{2in} \\ \hline
                \end{tabular}
                \vfill
    \end{enumerate}

\newpage
%----------------------------------------------------------------------
\item (8 points)
    Find the formula for $1+$ $\displaystyle\sum_{j=1}^n j!j $, and show work
    proving the formula is correct using induction.

    \hfill
    \begin{tabular}{|l|c|}
        \hline
        Formula: & \hspace{2in} \\ \hline
    \end{tabular}
    \vfill\vfill

%----------------------------------------------------------------------
\item (8 points)
    Indicate for each of the following pairs of expressions $(f(n), g(n))$,
    whether $f(n)$ is $O$, $\Omega$, or $\Theta$ of $g(n)$.  Prove your answers
    to the first two items, but just GIVE an answer to the last two.

    \begin{enumerate}
        \item $\displaystyle f(n) = 4^{\log_{4} n}$ and $g(n) = 2n+1$.
                \hfill
                \begin{tabular}{|l|c|}
                    \hline
                    Answer for (\theenumii): & $\quad f(n) \quad\qquad g(n)\quad$ \\ \hline
                \end{tabular}
                \vfill

        \item $\displaystyle f(n) = n^2$ and $g(n) = (\sqrt{2})^{\log_2 n}$.
                \hfill
                \begin{tabular}{|l|c|}
                    \hline
                    Answer for (\theenumii): & $\quad f(n) \quad\qquad g(n)\quad$ \\ \hline
                \end{tabular}
                \vfill

\newpage

        \item $f(n) = \log_2(n!)$ and $g(n) = n \log_2 n$.

                \hfill
                \begin{tabular}{|l|c|}
                    \hline
                    Answer for (\theenumii): & $\quad f(n) \quad\qquad g(n)\quad$ \\ \hline
                \end{tabular}
                \vfill

        \item $f(n) = n^k$ and $g(n) = c^n$ where $k$ and $c$ are constants and $c > 1$.

                \hfill
                \begin{tabular}{|l|c|}
                    \hline
                    Answer for (\theenumii): & $\quad f(n) \quad\qquad g(n)\quad$ \\ \hline
                \end{tabular}
                \vfill

    \end{enumerate}

%----------------------------------------------------------------------
\item (9 points)
    Solve the following recurrence relations for integer $n$. If no solution
    exists, please explain the result.

    \begin{enumerate}
        \item $T(n) = T(\frac{n}{2}) + 5$, $T(1) = 1$; assume $n$ is a power of 2.

                \hfill
                \begin{tabular}{|l|c|}
                    \hline
                    Answer for (\theenumii): & \hspace{2in} \\ \hline
                \end{tabular}
                \vfill


        \item $T(n) = T(n-1) + \frac{1}{n}$, $T(0) = 0$.
                \hfill
                \begin{tabular}{|l|c|}
                    \hline
                    Answer for (\theenumii): & \hspace{2in} \\ \hline
                \end{tabular}
                \vfill

        \item Prove that your answer to part (a) is correct using induction.
            \vfill\vfill

\end{enumerate}

\newpage
%----------------------------------------------------------------------
\item (10 points)
    Suppose function call parameter passing costs constant time, independent of
    the size of the structure being passed.

    \begin{enumerate}
        \item Give a recurrence for worst case running time of the recursive
            Binary Search function in terms of $n$, the size of the search
            array. Assume $n$ is a power of 2. Solve the recurrence.

            \vfill\vfill

            \hfill
            \begin{tabular}{|l|c|}
                \hline
                Recurrence: & \hspace{2in} \\ \hline
                Base case: & \\ \hline
                Recurrence Solution: &  \\ \hline
            \end{tabular}
            \vfill

        \item Give a recurrence for worst case running time of the recursive
            Merge Sort function in terms of $n$, the size of the array being
            sorted. Solve the recurrence.

            \vfill\vfill

            \hfill
            \begin{tabular}{|l|c|}
                \hline
                Recurrence: & \hspace{2in} \\ \hline
                Base case: & \\ \hline
                Running Time: &  \\ \hline
            \end{tabular}
            \vfill

    \end{enumerate}


\newpage
\item (10 points)
    Consider the pseudocode function below.
\begin{verbatim}
derp(x, n)
     if (n == 0)
         return 1;
     if (n % 2 == 0)
         return derp(x^2, n/2);
     return x * derp(x^2, (n-1)/2);
\end{verbatim}

    \begin{enumerate}
        \item What is the output when passed the following parameters: $x=2$,
            $n=12$? Show your work (activation diagram or similar).

            \hfill
            \begin{tabular}{|l|c|}
                \hline
                Answer for (\theenumii): & \hspace{2in} \\ \hline
            \end{tabular}
            \vfill

        \item Briefly describe what this function is doing.

            \vfill

        \item Write a recurrence that models the running time of this function.
            Assume checks, returns, and arithmetic are constant time, but be
            sure to evaluate all function calls. \emph{Hint: what is the
            \emph{most} $n$ could be at each level of the recurrence?}

            \vfill

        \item Solve the above recurrence for the running time of this function.

            \vfill

    \end{enumerate}

\end{problems}

\newpage
%----------------------------------------------------------------------
Blank sheet for extra work.


\end{document}
