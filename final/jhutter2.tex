\documentclass{article}
\usepackage{listings}
\usepackage{graphicx}
\usepackage{color}

\definecolor{mygreen}{rgb}{0,0.6,0}
\definecolor{mygray}{rgb}{0.5,0.5,0.5}
\definecolor{mymauve}{rgb}{0.58,0,0.82}

\lstset{ %
  backgroundcolor=\color{white},   % choose the background color; you must add \usepackage{color} or \usepackage{xcolor}; should come as last argument
  basicstyle=\footnotesize,        % the size of the fonts that are used for the code
  breakatwhitespace=false,         % sets if automatic breaks should only happen at whitespace
  breaklines=true,                 % sets automatic line breaking
  captionpos=b,                    % sets the caption-position to bottom
  commentstyle=\color{mygreen},    % comment style
  deletekeywords={...},            % if you want to delete keywords from the given language
  escapeinside={\%*}{*)},          % if you want to add LaTeX within your code
  extendedchars=true,              % lets you use non-ASCII characters; for 8-bits encodings only, does not work with UTF-8
  frame=single,	                   % adds a frame around the code
  keepspaces=true,                 % keeps spaces in text, useful for keeping indentation of code (possibly needs columns=flexible)
  keywordstyle=\color{blue},       % keyword style
  language=Octave,                 % the language of the code
  morekeywords={*,...},           % if you want to add more keywords to the set
  numbers=left,                    % where to put the line-numbers; possible values are (none, left, right)
  numbersep=5pt,                   % how far the line-numbers are from the code
  numberstyle=\tiny\color{mygray}, % the style that is used for the line-numbers
  rulecolor=\color{black},         % if not set, the frame-color may be changed on line-breaks within not-black text (e.g. comments (green here))
  showspaces=false,                % show spaces everywhere adding particular underscores; it overrides 'showstringspaces'
  showstringspaces=false,          % underline spaces within strings only
  showtabs=false,                  % show tabs within strings adding particular underscores
  stepnumber=2,                    % the step between two line-numbers. If it's 1, each line will be numbered
  stringstyle=\color{mymauve},     % string literal style
  tabsize=2,	                   % sets default tabsize to 2 spaces
  title=\lstname                   % show the filename of files included with \lstinputlisting; also try caption instead of title
}

\author{Jacob Hutter}
\title{ECE 311 Lab 7}

\begin{document}
\maketitle

\color{red}
\underline{\textbf{Sakanaya}}
\color{black}
\begin{figure}[H]
\lstinputlisting[language=Matlab]{Sakanaya.m}
\includegraphics[scale=.5]{Sakanaya1}
From the plot we see that there are tones at $+/- 10,12,22,30 and 45 Hz$ so a total of 10 tones.
\end{figure}


\begin{figure}[H]
  \color{red}
  \underline{\textbf{Chipotle}}
  \color{black}
\lstinputlisting[language=Matlab]{Chipotle.m}
\includegraphics[scale=.5]{Chipotle1}
\end{figure}

\begin{figure}[H]
\includegraphics[scale=.5]{Chipotle2}
\includegraphics[scale=.5]{Chipotle3}
\includegraphics[scale=.5]{Chipotle4}
After upsampling, we see the frequency effect because of the time and frequency scaling property of the Fourier transform. The maximum we could have downsampled in order to avoid aliasing is 4.
\end{figure}


\begin{figure}[H]
  \color{red}
  \underline{\textbf{Legends}}
  \color{black}
\lstinputlisting[language=Matlab]{Legends.m}
\includegraphics[scale=.5]{Legends1}
\end{figure}

\begin{figure}[H]
\includegraphics[scale=.5]{Legends2}
\includegraphics[scale=.5]{legends3}
\includegraphics[scale=.5]{Legends4}
After filtering the correct frequencies seen from sig in frequency domain. We notice that there are 10 occurences of sig in the time domain of x.
\end{figure}


\begin{figure}[H]
  \color{red}
  \underline{\textbf{Blackdog}}
  \color{black}
\lstinputlisting[language=Matlab]{Blackdog.m}
\includegraphics[scale=.5]{Blackdog1}
\end{figure}

\begin{figure}[H]
\includegraphics[scale = .5]{Blackdog2}
\includegraphics[scale = .5]{Blackdog3}
After listening to the sound, we notice a linear increase in frequency that is seen by the spectrogram. The friend incorrectly downsampled because he did not low pass filter after downsampling which introduced copies into the mix.
Therefore, we heard a shortened sound and a reverse version. After we lowpass filtered, We got the correct shortened version of the original signal.
\end{figure}

\end{document}
