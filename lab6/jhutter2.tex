\documentclass{article}
\usepackage{listings}
\usepackage{graphicx}
\usepackage{color}

\definecolor{mygreen}{rgb}{0,0.6,0}
\definecolor{mygray}{rgb}{0.5,0.5,0.5}
\definecolor{mymauve}{rgb}{0.58,0,0.82}

\lstset{ %
  backgroundcolor=\color{white},   % choose the background color; you must add \usepackage{color} or \usepackage{xcolor}; should come as last argument
  basicstyle=\footnotesize,        % the size of the fonts that are used for the code
  breakatwhitespace=false,         % sets if automatic breaks should only happen at whitespace
  breaklines=true,                 % sets automatic line breaking
  captionpos=b,                    % sets the caption-position to bottom
  commentstyle=\color{mygreen},    % comment style
  deletekeywords={...},            % if you want to delete keywords from the given language
  escapeinside={\%*}{*)},          % if you want to add LaTeX within your code
  extendedchars=true,              % lets you use non-ASCII characters; for 8-bits encodings only, does not work with UTF-8
  frame=single,	                   % adds a frame around the code
  keepspaces=true,                 % keeps spaces in text, useful for keeping indentation of code (possibly needs columns=flexible)
  keywordstyle=\color{blue},       % keyword style
  language=Octave,                 % the language of the code
  morekeywords={*,...},           % if you want to add more keywords to the set
  numbers=left,                    % where to put the line-numbers; possible values are (none, left, right)
  numbersep=5pt,                   % how far the line-numbers are from the code
  numberstyle=\tiny\color{mygray}, % the style that is used for the line-numbers
  rulecolor=\color{black},         % if not set, the frame-color may be changed on line-breaks within not-black text (e.g. comments (green here))
  showspaces=false,                % show spaces everywhere adding particular underscores; it overrides 'showstringspaces'
  showstringspaces=false,          % underline spaces within strings only
  showtabs=false,                  % show tabs within strings adding particular underscores
  stepnumber=2,                    % the step between two line-numbers. If it's 1, each line will be numbered
  stringstyle=\color{mymauve},     % string literal style
  tabsize=2,	                   % sets default tabsize to 2 spaces
  title=\lstname                   % show the filename of files included with \lstinputlisting; also try caption instead of title
}

\author{Jacob Hutter}
\title{ECE 311 Lab 6}

\begin{document}
\maketitle

\begin{figure}[H]
\color{red}
\underline{\textbf{Report Item 1}}
\color{black}
\lstinputlisting[language=Matlab]{report1.m}
\includegraphics[scale=.5]{report1_1}
\end{figure}

\begin{figure}[H]
\includegraphics[scale=.5]{report1_2}
\end{figure}

\begin{figure}[H]
\color{red}
\underline{\textbf{Report Item 2}}
\color{black}
$A = U \Sigma V^H $
\\$A^HA = V\Sigma^HU^HU \Sigma V^H $
\\$= V \Sigma^H \Sigma V^H$
\\ $A^HAV= V \Sigma^H \Sigma = V \Sigma^2$
\end{figure}

\begin{figure}[H]
\color{red}
\underline{\textbf{Report Item 3}}
\color{black}
\lstinputlisting[language=Matlab]{report2.m}
\includegraphics[scale = .5]{report3_1}
\\As you can see the resulting matricies are essentially all zeros.
\end{figure}

\begin{figure}[H]
\color{red}
\underline{\textbf{Report Item 4}}
\color{black}
\lstinputlisting[language=Matlab]{report4.m}
\end{figure}

\begin{figure}[H]
\includegraphics[scale =.4]{report4_1}
\includegraphics[scale =.4]{report4_2}
\\Above you can see that we see a cosine waveform in the real plot and a sine in the imaginary. As the row number goes up, we can see in the third and fourth plots that the frequency increases
 and then decreases.
\end{figure}

\begin{figure}[H]
\color{red}
\underline{\textbf{Report Item 5}}
\color{black}
\lstinputlisting[language=Matlab]{report5.m}
\includegraphics[scale =.5]{report5_1}
\\You can see by the plot I have included that the resulting relationship shows that they are orthonormal and orthoganol.
\end{figure}


\begin{figure}[H]
  \color{red}
  \underline{\textbf{Report Item 6}}
  \color{black}
\lstinputlisting[language=Matlab]{report6.m}
\lstinputlisting[language=Matlab]{compMeanVec.m}
\includegraphics[scale =.5]{report6_1}
\\After the mean vector has been computed, the resulting image is the average of all of the images inside yalefaces.
\end{figure}

\begin{figure}[H]
  \color{red}
  \underline{\textbf{Report Item 7}}
  \color{black}
\lstinputlisting[language=Matlab]{report7.m}
\end{figure}

\begin{figure}[H]
\includegraphics[scale =.5]{report7_eig}
\\By the eigenvalue plot, the values sharply die off after the 20th value or so.
\end{figure}


\begin{figure}[H]
\includegraphics[scale =.4]{report7_1}
\includegraphics[scale =.4]{report7_2}
\end{figure}

\begin{figure}[H]
  \includegraphics[scale =.4]{report7_3}
  \includegraphics[scale =.4]{report7_4}
  \\The early eigen vectors look like ghosts or multiple faces overlayed.
\end{figure}

\begin{figure}[H]
\includegraphics[scale =.4]{report7_5}
\includegraphics[scale =.4]{report7_6}
\\The later eigen vectors past 50 look very staticky and are less discernible as faces.
\end{figure}

\begin{figure}[H]
  \color{red}
  \underline{\textbf{Report Item 8}}
  \color{black}
  \\ Below are the implementations for PCAtransform and invPCAtransform
\lstinputlisting[language=Matlab]{PCAtransform.m}
\lstinputlisting[language=Matlab]{invPCAtransform.m}
\end{figure}

\begin{figure}[H]
  \color{red}
  \underline{\textbf{Report Item 9}}
  \color{black}
\lstinputlisting[language=Matlab]{report8.m}
\includegraphics[scale =.5]{report8}
\end{figure}

\end{document}
